\documentclass[11pt]{article}
\usepackage[left=2cm, right=2cm, top=2.5cm, bottom=2.5cm]{geometry}
\usepackage[MeX]{polski}
\usepackage{graphicx}
\usepackage{enumerate}
\title{KURCZAK Z RYŻEM I PARMEZANEM}
\author{Eryk Żabiński}
\date{Październik 2022}
\begin{document}
\maketitle
\begin{figure}[h]
\centering
\includegraphics[scale=0.4]{kurczak.jpg}
\end{figure}
\newpage
\section{Składniki}
\begin{itemize}
\item 500 g filetów kurczaka
\item 1/2 łyżeczki sproszkowanego czosnku 
\item 200 g ryżu (np. basmati)
\item 2 łyżki oleju
\item 2 ząbki czosnku
\item 2 łyżki masła
\item szczypta płatków chili
\item 1/3 szklanki białego wina
\item 3 szklanki bulionu
\item 1 szklanka drobno startego parmezanu lub grana padano
\item posiekana natka pietruszki
\end{itemize}
\section{Przygotowanie}
\begin{enumerate}
\item Kurczaka pokroić na 6 mniejszych porcji - filetów podobnej wielkości i grubości. Doprawić je solą, pieprzem oraz sproszkowanym czosnkiem.
\item Rozgrzać olej na dużej patelni (takiej z pokrywą), włożyć kurczaka i obsmażyć na złoto z dwóch stron (w sumie ok. 6 minut smażenia, w zależności od grubości filetów). Wyjąć na talerz.
\item Na tę samą patelnię wrzucić obrany i stary czosnek, dodać masło oraz płatki chili i mieszając chwilę podsmażać.
\item Wlać wino i odparować połowę ilości. Odłożyć 2 łyżki otrzymanego płynu a do reszty na patelni wsypać suchy ryż. Mieszając smażyć przez ok. 1 - 2 minuty aż ryż wchłonie cały płyn.
\item Wówczas dodać bulion, wymieszać i zagotować. Przykryć pokrywą i gotować przez ok. 10 minut na umiarkowanym ogniu.
\item Zdjąć pokrywę, wymieszać ryż, wyrównać powierzchnię, posypać parmezanem, ułożyć filety kurczaka, polać je odłożonym sosem i podgrzewać na małym ogniu bez przykrycia przez około 5 minut.
\item Odstawić z ognia i posypać natką pietruszki.
\end{enumerate}
\end{document}
